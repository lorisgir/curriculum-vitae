\sectionTitle{Esperienza Lavorativa}{\faSuitcase}
%\renewcommand{\labelitemi}{$\bullet$}
\begin{experiences}

        \experience
{Giugno 2025}   {CTO e Co-Fondatore}{Lumen Labs - Talkkit, Torino TO}
{Gennaio 2024} {
	\begin{itemize}
	\item Co-fondatore e CTO di una startup innovativa nel settore Ed-Tech. Il nostro prodotto, \textbf{Talkkit}, è un'applicazione per l'apprendimento delle lingue che permette di conversare in modo naturale con un \textbf{avatar virtuale intelligente}. L'app è stata sviluppata in \textbf{Flutter}, con distribuzione su \textbf{iOS e Android}, e integra un modulo in \textbf{Unity} per la gestione dell’avatar 3D, comprensivo di modelli animati e sincronizzazione labiale.

	In qualità di CTO, mi sono occupato di tutte le fasi del ciclo di vita del prodotto: dall'ideazione e ricerca iniziale, alla progettazione, sviluppo, testing e distribuzione in ambienti \textbf{cloud Azure}. Il backend è stato realizzato in \textbf{FastAPI (Python)}.

	Come Co-Fondatore, oltre all’aspetto tecnico, ho ricoperto un ruolo attivo nella gestione trasversale della startup, seguendo attività operative, amministrative e strategiche, come il marketing, la comunicazione con potenziali investitori, e la definizione del piano di sviluppo aziendale.
\end{itemize}

}
{Flutter, FastAPI, Unity, Azure}
\emptySeparator
\emptySeparator
\experience
{Presente}   {Full Stack Developer}{DoInn SRL, Mondovì CN}
{Aprile 2023} {
	\begin{itemize}
		\item Coinvolto in tutte le fasi del ciclo di sviluppo software per soluzioni web-based custom richieste da clienti, tra cui \textbf{Banco Azzaglio SpA}. Mi occupo di raccolta requisiti, analisi, progettazione, sviluppo, test e rilascio, sia in autonomia che all'interno di team come \textit{Team Leader}.
		
		Ho contribuito a diversi progetti significativi:
		
		\begin{enumerate}
		\item \textbf{Piattaforma di Welfare Aziendale}, sviluppata come ecosistema composto da \textbf{quattro interfacce distinte} per i diversi attori coinvolti nel processo: \textit{Dipendenti}, \textit{Aziende}, \textit{Esercenti (Negozi/Punti Vendita)} e \textit{Admin}. Il sistema include:

\begin{itemize}
	\item Un'interfaccia \textbf{Next.js} per il pannello \textit{Admin}, dedicata alla gestione centrale del sistema (approvazioni, monitoraggio, censimenti, bandi, reportistica).
	\item Un'interfaccia \textbf{Next.js} per le \textit{Aziende}, con funzionalità di onboarding dipendenti, gestione fondi welfare e visualizzazione delle spese.
	\item Un'app \textbf{React Native} per gli \textit{Utenti finali (dipendenti)}, per la consultazione del credito disponibile, richiesta e utilizzo di voucher presso gli esercenti, ricerca di bandi e benefit.
	\item Un'app \textbf{React Native} per gli \textit{Esercenti}, per la gestione e riscossione dei voucher e la consultazione dello storico transazioni.
\end{itemize}

Il tutto è orchestrato tramite un backend centralizzato sviluppato in \textbf{Laravel}, che espone API REST e gestisce autenticazione, autorizzazioni, logiche di business, e integrazioni con sistemi esterni per l’emissione dei voucher e la gestione di bandi pubblici. 

			
		\item \textbf{Piattaforma di Sound Therapy}, sviluppata per supportare percorsi terapeutici personalizzati. Permette a operatori sanitari di eseguire programmi di valutazione del benessere anche tramite l'utilizzo di un dispositivo elettromagnetico. La piattaforma è strutturata in modalità \textbf{multi-tenant}, con accessi distinti per:
\begin{itemize}
	\item \textit{Utenti finali}, che accedono a un portale con funzionalità e-commerce per l'acquisto di prodotti e servizi legati alla terapia.
	\item \textit{Dottori}, che possono gestire i propri pazienti ed eseguire i programmi di valutazione.
	\item \textit{Amministratori}, che utilizzano un CMS sviluppato ad hoc per la creazione  dei programmi di valutazione e per la gestione di configurazioni varie. 
\end{itemize}
\textit{Stack: React, Laravel, FastAPI}.

\item \textbf{Chatbot RAG} (Retrieval-Augmented Generation), integrato con un archivio documentale interno all’intranet bancaria. Il sistema consente il recupero e la consultazione di informazioni contenute in PDF , sfruttando modelli GPT e un motore vettoriale basato su \textbf{QDrant}, per offrire risposte precise e contestualizzate agli operatori. \textit{Stack: Sencha Ext JS, FastAPI, QDrant}.

\item \textbf{Sistema per la gestione delle cartolarizzazioni} in ambito bancario, sviluppato per l’utilizzo interno alla rete intranet dell’istituto. Il sistema consente di gestire il ciclo di vita delle operazioni di cartolarizzazione con attenzione alla \textbf{storicizzazione dei dati}, alla \textbf{validazione delle informazioni} inserite dagli operatori, e alla \textbf{generazione di report} ufficiali per la comunicazione verso altri enti bancari e autorità. \textit{Stack: Sencha Ext JS, Laravel}.

\item \textbf{Piattaforma per il noleggio di materiale hardware}, progettata come sistema \textbf{multi-tenant} per aziende fornitrici di attrezzature tecnologiche. Ogni azienda ha accesso a un pannello dedicato con cui può gestire i propri clienti, configurare prodotti, definire listini e generare ordini di noleggio. \textit{Stack: React, Laravel}.

		\end{enumerate}
	\end{itemize}
        }
	{NextJS, Sencha Ext JS, Laravel, React, FastAPI, QDrant}
	\emptySeparator
	\emptySeparator
	\experience
	{Gennaio 2023}   {Tesi di Laurea - Data Scientist}{Alten Italia S.p.A., Torino TO}
	{Settembre 2022} {
		\begin{itemize}
			\item Studio e analisi di dati Time Series per l'Anomaly Detection.
			\item Applicazioni di metodi Machine Learning.
		\end{itemize}
	}
	{Machine Learning, Deep Learning, Anomaly Detection, Time Series}
	\emptySeparator
	\emptySeparator
	\experience
	{Settembre 2022}   {Full Stack Developer}{Sguang Informatica SRL, San Rocco CN}
	{Dicembre 2019} {
		\begin{itemize}
			\item Sviluppo Web di software gestionali con funzionalità e specifiche fornite da aziende clienti utilizzando \textbf{Javascript} (Vue.JS) per front-end e \textbf{Python} (Django) per back-end. Coinvolgimento in tutte le fasi di progettazione e sviluppo: dalla raccolta dati fino al deploy su server. 
            
            In generale questi gestionali ricadevano tutti nella gestione dei clienti e dei loro appuntamenti, ordini di vendita, preventivi, noleggi, interventi o manutenzioni.
			\item Gestione Hosting Web, Domini e VPS attraverso i servizi di Aruba, Register e Netsons.
			\item Sviluppo Web di piccoli applicativi con PHP.
			\item Sviluppo siti web in Wordpress.                         
		\end{itemize}
	}
	{Vue, Django, Wordpress, XAMPP}
	
	 
	
\end{experiences}
\sectionTitle{Progetti Personali}{\faLaptop}

\begin{projects}
	\project
	{Anomaly Detection Non Supervisionato per un industria Manifatturiera}{2023}
	{Tesi di ricerca per la Laurea Magistrale. 
		 
		\github{lorisgir/TESI-MAGISTRALE}. }
	{Machine Learning, Deep Learning, Time Series, Model Selection, Anomaly Detection}
	
	\project
	{Booksriver}{2022}
	{Booksriver è una applicazione mobile. Mette a disposizione un’ampia libreria di libri in cui l’utente può navigare, indicare il suo progresso nella lettura, scrivere recensioni e seguire altri utenti. L'applicazione e' stata sviluppata in tre varianti: Nativa Android utilizzando Java e XML, Nativa Android utilizzando Kotlin e Compose, multi piattaforma utilizzando Flutter. 
		 
		Il backend è stato progettato attraverso il framework Spring Boot secondo un'architettura a microservizi, tra cui anche quelli di comunicazione e coordinamento come RabbitMQ, EUREKA e Gateway. Per il deploy si e' fatto l'uso di Docker-Compose e le CI di GitLab. 
		 
		\github{lorisgir/Booksriver}. }
	{Spring, Java, Kotlin, Dart, Flutter, Compose, Docker, Microservizi}
	
	 
	\project
	{Progettazione e sviluppo di un software per la gestione del Piano Assistenziale Individualizzato}{2020}
	{Progetto di Tesi per la Laurea Triennale. Il software sviluppato offre un supporto  digitale agli operatori sanitari delle RSA andando ad informatizzare la compilazione dei documenti del Piano Assistenziale Individualizzato.}
	{Vue, Django, PostgreSQL}
					
		
	
	
	
	 
	
\end{projects}